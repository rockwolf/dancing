\subsection{Counting}
1 2 3 and 5 6 7 and

\subsection{Basic 1}
Naturally, like walking, loose shoulders.
\begin{table}[H]
\centering
\begin{tabular}{cc}
  \toprule
  \textbf{Count} & \textbf{Movement}\\
  \cmidrule(lr){1-1}\cmidrule(lr){2-2}
  1 & LF forward\\
  2 & RF in place\\
  3 & Pull back LF\\
  and &\\
  5 & RF backward\\
  6 & LF in place\\
  7 & Pull back RF\\
  and &\\
  \bottomrule
\end{tabular}
\label{basic1_feet}
\caption{basic 1, feet}
\end{table}

Move you hands in an internity symbol like movement in front of you.
\begin{table}[H]
\centering
\begin{tabular}{cc}
  \toprule
  \textbf{Count} & \textbf{Movement}\\
  \cmidrule(lr){1-1}\cmidrule(lr){2-2}
  1 & RH\\
  2 & LH\\
  3 & RH\\
  and & \\
  5 & LH\\
  6 & RH\\
  7 & LH\\
  and & \\
  \bottomrule
\end{tabular}
\label{basic1_hands}
\caption{basic 1, hands}
\end{table}

\subsection{Basic 2}
Hands against each other, called open position.\\
Let her feel we are going to basic 2 by pulling the hands to the left a bit.\\
Start to the left.\\
\begin{table}[H]
\centering
\begin{tabular}{cc}
  \toprule
  \textbf{Count} & \textbf{Movement}\\
  \cmidrule(lr){1-1}\cmidrule(lr){2-2}
  1 & LF to the left\\
  2 & RF in place\\
  3 & Pull back LF\\
  and & \\
  5 & RF to the right\\
  6 & LF in place\\
  7 & Pull back RF\\
  and & \\
  \bottomrule
\end{tabular}
\label{basic2_steps}
\caption{basic 2, steps}
\end{table}

\subsection{Basic 3}
Left hand pushes her richt hand away, to make it clear that we are going to basic 3.\\
\begin{table}[H]
\centering
\begin{tabular}{cc}
  \toprule
  \textbf{Count} & \textbf{Movement}\\
  \cmidrule(lr){1-1}\cmidrule(lr){2-2}
  1 & LF behind RF and turn torso and LS backwards\\
  2 & weight on RF to prepare to turn to the right\\
  3 & Put LF back next to your RF\\
  and & \\
  5 & RF behind LF and turn torso and RS backwards\\
  6 & weight on LF to prepare to turn to the left\\
  7 & Put RF back next to your LF\\
  and & \\
  \bottomrule
\end{tabular}
\label{basic3_steps}
\caption{basic 3, steps}
\end{table}

\subsection{Dile que no}
\begin{table}[H]
\centering
\begin{tabular}{cc}
  \toprule
  \textbf{Count} & \textbf{Movement}\\
  \cmidrule(lr){1-1}\cmidrule(lr){2-2}
  1 & LF forward, start bringing her hand upwards\\
  2 & RF forward too, her hand should be up now\\
  3 & weight on LF to complete basic 1\\
  and & \\
  5 & LF sidesteps left so she can pass\\
  6 & she turns inward, hold palm upward as her support, you put your RF next to LF\\
  7 & Put weight back on LF, you are at a 90 degree angle with her\\
  and & \\
  1 & LF forward\\
  2 & weight on RF, while she starts turning back the other way\\
  3 & Put LF back again\\
  and & \\
  5 & RF back, while she turns in front of you\\
  6 & LF back, make way for her to turn\\
  7 & Put RF back next to LF, while you turn towards her\\
  and & \\
  \bottomrule
\end{tabular}
\label{dileequeno_steps}
\caption{dile que no, steps}
\end{table}

Make sure that you start, by pointing your LF more to the outside when you do this move.\\
If the foot is already pointing 90 degrees to the left, it will make turning towards her easier.

\subsection{Dile que no con vuelta}
Same as dile que no, but instead of just moving her back in place, hold your hand up and make her do an extra turn counterclockwise,
during her return to your left side.

\subsection{Right turn for men}
\begin{table}[H]
\centering
\begin{tabular}{cc}
  \toprule
  \textbf{Count} & \textbf{Movement}\\
  \cmidrule(lr){1-1}\cmidrule(lr){2-2}
  1 & LF forward, start bringing her hand upwards\\
  2 & RF forward too, her hand should be up now\\
  3 & weight on LF to complete basic 1\\
  and & \\
  5 & LF sidesteps left so she can pass\\
  6 & she turns inward, hold palm upward as her support, you put your RF next to LF\\
  7 & Put weight back on LF, turn towards her\\
  and & \\
  1 & Take her RH and bring it up, LF forward\\
  2 & Step under her arm and start turning CCW, RF forward\\
  3 & Put weight on LF again, you are now turned completely\\
  and & \\
  5 & RF back\\
  6 & LF in place\\
  7 & Pull back RF\\
  and & \\
  \bottomrule
\end{tabular}
\label{right_turn_for_men_steps}
\caption{Right turn for men, steps}
\end{table}

\subsection{Right turn for both}
\begin{table}[H]
\centering
\begin{tabular}{cc}
  \toprule
  \textbf{Count} & \textbf{Movement}\\
  \cmidrule(lr){1-1}\cmidrule(lr){2-2}
  1 & LF forward\\
  2 & RF in place\\
  3 & Pull back LF\\
  and & \\
  5 & RF backward, start bringing her right hand upward\\
  6 & LF in place\\
  7 & Pull back LF\\
  and & \\
  1 & Start turn to the right with LF, under her lifted arm\\
  2 & RF clockwise behind LF\\
  3 & Finish turn and stand front-facing again, push her LF back CCW\\
  and & \\
  5 & RF backward, while she turns CCW under your RA\\
  6 & LF in place, while she finishes her turn\\
  7 & Pull back RF\\
  and & \\
  \bottomrule
\end{tabular}
\label{right_turn_for_both_steps}
\caption{right turn for both, steps}
\end{table}

\subsection{Holding, open vs. closed}
\subsubsection{Open}
Man has his hands at the bottom, with palms facing up.\\
Just a few fingers hold her hands.

\subsubsection{Closed}
LA in a V-shape.\\
Your thumb in the palm of her hand.\\
Your fingers around her wrist.\\
Keep your right hand high on her upper back.

\subsubsection{Closed position turning}
RF to the right, left foot makes room.
\begin{table}[H]
\centering
\begin{tabular}{cc}
  \toprule
  \textbf{Count} & \textbf{Movement}\\
  \cmidrule(lr){1-1}\cmidrule(lr){2-2}
  1 & RF forward to the right\\
  2 & LF make room, so she can pass\\
  3 & Weight to RF\\
  and & \\
  5 & RF backward, she passes\\
  6 & LF turns to the left, to follow her\\
  7 & Follow with RF and go back to basic 1\\
  and & \\
  \bottomrule
\end{tabular}
\label{closed_position_turning}
\caption{closed position turning, steps}
\end{table}

\subsection{Basic 4}
Put some pressure on her hands to the right, will signal the start of basic 4.\\
\begin{table}[H]
\centering
\begin{tabular}{cc}
  \toprule
  \textbf{Count} & \textbf{Movement}\\
  \cmidrule(lr){1-1}\cmidrule(lr){2-2}
  1 & LF to the right, before RF and put it next to it\\
  2 & Put RF behind LF\\
  3 & Put weight back on LF\\
  and & \\
  5 & RF to the left, before LF and put it next to it\\
  6 & Put LF behind RF\\
  7 & Put weight back on LF\\
  and & \\
  \bottomrule
\end{tabular}
\label{basic4_steps}
\caption{basic 4, steps}
\end{table}

\subsection{Suzy Q}
\subsubsection{Open}
You start by bringing your right hand down and your left hand up.\\
This is a salsa shines figure, which is partially used in basic 4.\\
Count \textit{SU-ZY-Q} for the steps.\\
\begin{table}[H]
\centering
\begin{tabular}{cc}
  \toprule
  \textbf{Count} & \textbf{Movement}\\
  \cmidrule(lr){1-1}\cmidrule(lr){2-2}
  1 & LF to the right, before RF and put it next to it\\
  2 & Weight on RF\\
  3 & Weight back on LF\\
  and & \\
  5 & RF to the left, before LF and put it next to it\\
  6 & Weight on LF\\
  7 & Weight back on RF\\
  and & \\
  \multicolumn{2}{c}{Repeat 2 times}\\
  \multicolumn{2}{c}{Return to basic 4 or basic 1}\\
  \bottomrule
\end{tabular}
\label{suzyq_steps}
\caption{Suzy Q, steps}
\end{table}

Note: To return to basic 1, put some pressure on her hands to push her back.

\subsubsection{Closed}
Count 5 6 7 and...\\
Now start by putting your LF to the right.\\
Pull her RA down and keep her LH high.\\
This way, she falls into the movement.

\subsection{Left turn for woman}
\begin{table}[H]
\centering
\begin{tabular}{cc}
  \toprule
  \textbf{Count} & \textbf{Movement}\\
  \cmidrule(lr){1-1}\cmidrule(lr){2-2}
  1 & LF forward\\
  2 & RF in place\\
  3 & Pull back LF\\
  and &\\
  5 & RF backward, start lifting her LH\\
  6 & LF in place, LH should be lifted\\
  7 & Pull back RF\\
  and &\\
  1 & LF to the left, she starts turning CCW in front of you\\
  2 & Pull RF next to your LF, so you stand to her left\\
  3 & Weight back on LF\\
  and &\\
  \multicolumn{2}{c}{Rotate towards her and finish with second part of basic 1}\\
  \multicolumn{2}{c}{Return to basic 1}\\
  and &\\
  \bottomrule
\end{tabular}
\label{left_turn_for_woman_steps}
\caption{Left turn for woman, steps}
\end{table}

\subsection{Enchufla}
\begin{table}[H]
\centering
\begin{tabular}{cc}
  \toprule
  \textbf{Count} & \textbf{Movement}\\
  \cmidrule(lr){1-1}\cmidrule(lr){2-2}
  1 & LF forward\\
  2 & RF in place\\
  3 & Pull back LF\\
  and &\\
  5 & RF backward, start lifting both her arms\\
  6 & LF in place, her arms should be lifted\\
  7 & Pull back RF\\
  and &\\
  1 & LF to the left, she starts a turn CCW\\
  2 & Pull RF next to your LF, so you stand to her left\\
  3 & Turn towards her at a 90 degree angle, she finishes her turn\\
  and &\\
  \multicolumn{2}{c}{Rotate towards her and finish with second part of basic 1}\\
  \multicolumn{2}{c}{Do a dile que no}\\
  \multicolumn{2}{c}{Return to basic 1}\\
  \bottomrule
\end{tabular}
\label{left_turn_for_woman_steps}
\caption{Left turn for woman, steps}
\end{table}

\subsection{Enchufla doble}
Same as enchufla.
After you block her, she turns back.
She turns back.
Now finish like with enchufla.
So you just have 1 extra turn with shoulder block in there.
The rest is the same a with a regular enchufla.
\begin{table}[H]
\centering
\begin{tabular}{cc}
  \toprule
  \textbf{Count} & \textbf{Movement}\\
  \cmidrule(lr){1-1}\cmidrule(lr){2-2}
  1 & LF forward\\
  2 & RF in place\\
  3 & Pull back LF\\
  and &\\
  5 & RF backward, start lifting both her arms\\
  6 & LF in place, her arms should be lifted\\
  7 & Pull back RF\\
  and &\\
  1 & LF to the left, she starts a turn CCW\\
  2 & Pull RF next to your LF, so you stand to her left\\
  3 & Turn towards her at a 90 degree angle, block her left shoulder\\
  and &\\
  5 & Let her turn back, start basic 3\\
  6 & She turns in front of you, still doing basic 3\\
  7 & She turned, end of first part of basic 3\\
  and &\\
  \multicolumn{2}{c}{She turns back again, while you still do basic 3}\\
  \multicolumn{2}{c}{Don't block her this time}\\
  \multicolumn{2}{c}{Rotate towards her and finish with second part of basic 1}\\
  \multicolumn{2}{c}{Do a dile que no}\\
  \multicolumn{2}{c}{Return to basic 1}\\
  \bottomrule
\end{tabular}
\label{left_turn_for_woman_steps}
\caption{Left turn for woman, steps}
\end{table}


\subsection{Twisted arm lock on woman}
\begin{table}[H]
\centering
\begin{tabular}{cc}
  \toprule
  \textbf{Count} & \textbf{Movement}\\
  \cmidrule(lr){1-1}\cmidrule(lr){2-2}
  1 & LF forward, lift her LH and lower her RH\\
  2 & RF in place, make sure her RH pinky is pointed upwards (cfr. wrist lock)\\
  3 & Pull back LF\\
  and &\\
  \multicolumn{2}{c}{Let her turn CCW, under your LH on 5 6 7...}\\
  \multicolumn{2}{c}{Step to the left and let her rotate out of her arm lock}\\
  \multicolumn{2}{c}{This rotation is a right turn for her}\\
  \multicolumn{2}{c}{After the rotation, guide her LH to your right temple}\\
  \multicolumn{2}{c}{Let the hand loose}\\
  \multicolumn{2}{c}{She puts her hands around your neck}\\
  \multicolumn{2}{c}{Your RH on her lower back on 5 6 7...}\\
  \multicolumn{2}{c}{Do a dile que no}\\
  \multicolumn{2}{c}{Return to basic 1}\\
  \bottomrule
\end{tabular}
\label{twisted_arm_lock_steps}
\caption{Twisted arm lock on woman, steps}
\end{table}

\subsection{Sombrero}
TBD
