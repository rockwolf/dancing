\subsection{Counting}
1 2 3 and 5 6 7 and

\subsection{Basic 1}
Naturally, like walking, loose shoulders.
\begin{table}[H]
\centering
\begin{tabular}{cc}
  \toprule
  \textbf{Count} & \textbf{Movement}\\
  \cmidrule(lr){1-1}\cmidrule(lr){2-2}
  1 & LF forward\\
  2 & RF in place\\
  3 & Pull back LF\\
  and &\\
  5 & RF backward\\
  6 & LF in place\\
  7 & Pull back RF\\
  and &\\
  \bottomrule
\end{tabular}
\label{basic1_feet}
\caption{basic 1, feet}
\end{table}

Move you hands in an internity symbol like movement in front of you.
\begin{table}[H]
\centering
\begin{tabular}{cc}
  \toprule
  \textbf{Count} & \textbf{Movement}\\
  \cmidrule(lr){1-1}\cmidrule(lr){2-2}
  1 & RH\\
  2 & LH\\
  3 & RH\\
  and & \\
  5 & LH\\
  6 & RH\\
  7 & LH\\
  and & \\
  \bottomrule
\end{tabular}
\label{basic1_hands}
\caption{basic 1, hands}
\end{table}

\subsection{Basic 2}

Hands against each other, called open position.\\
Let her feel we are going to basic 2 by pulling the hands to the left a bit.\\
Start to the left.
\begin{table}[H]
\centering
\begin{tabular}{cc}
  \toprule
  \textbf{Count} & \textbf{Movement}\\
  \cmidrule(lr){1-1}\cmidrule(lr){2-2}
  1 & LF to the left\\
  2 & RF in place\\
  3 & Pull back LF\\
  and & \\
  5 & RF to the right\\
  6 & LF in place\\
  7 & Pull back RF\\
  and & \\
  \bottomrule
\end{tabular}
\label{basic2_steps}
\caption{basic 2, steps}
\end{table}

\subsection{Basic 3}

Left hand pushes her richt hand away, to make it clear that we are going to basic 3.\\
\begin{table}[H]
\centering
\begin{tabular}{cc}
  \toprule
  \textbf{Count} & \textbf{Movement}\\
  \cmidrule(lr){1-1}\cmidrule(lr){2-2}
  1 & LF behind RF and turn torso and LS backwards\\
  2 & weight on RF to prepare to turn to the right\\
  3 & Put LF back next to your RF\\
  and & \\
  5 & RF behind LF and turn torso and RS backwards\\
  6 & weight on LF to prepare to turn to the left\\
  7 & Put RF back next to your LF\\
  and & \\
  \bottomrule
\end{tabular}
\label{basic3_steps}
\caption{basic 3, steps}
\end{table}

\subsection{Dile que no}
\begin{table}[H]
\centering
\begin{tabular}{cc}
  \toprule
  \textbf{Count} & \textbf{Movement}\\
  \cmidrule(lr){1-1}\cmidrule(lr){2-2}
  1 & LF forward, start bringing her hand upwards\\
  2 & RF forward too, her hand should be up now\\
  3 & weight on LF to complete basic 1\\
  and & \\
  5 & LF sidesteps left so she can pass\\
  6 & she turns inward, hold palm upward as her support, you put your RF next to LF\\
  7 & Put weight back on LF, you are at a 90 degree angle with her\\
  and & \\
  1 & LF forward\\
  2 & weight on RF, while she starts turning back the other way\\
  3 & Put LF back again\\
  and & \\
  5 & RF back, while she turns in front of you\\
  6 & LF back, make way for her to turn\\
  7 & Put RF back next to LF, while you turn towards her\\
  and & \\
  \bottomrule
\end{tabular}
\label{dileequeno_steps}
\caption{dile que no, steps}
\end{table}

\subsection{Dile que no con vuelta}
Same as dile que no, but instead of just moving her back in place, hold your hand up and make her do an extra turn counterclockwise,
during her return to your left side.

\subsection{Right turn for men}
\begin{table}[H]
\centering
\begin{tabular}{cc}
  \toprule
  \textbf{Count} & \textbf{Movement}\\
  \cmidrule(lr){1-1}\cmidrule(lr){2-2}
  1 & LF forward, start bringing her hand upwards\\
  2 & RF forward too, her hand should be up now\\
  3 & weight on LF to complete basic 1\\
  and & \\
  5 & LF sidesteps left so she can pass\\
  6 & she turns inward, hold palm upward as her support, you put your RF next to LF\\
  7 & Put weight back on LF, turn towards her\\
  and & \\
  1 & Take her RH and bring it up, LF forward\\
  2 & Step under her arm and start turning counterclockwise, RF forward\\
  3 & Put weight on LF again, you are now turned completely\\
  and & \\
  5 & RF back\\
  6 & LF in place\\
  7 & Pull back RF\\
  and & \\
  \bottomrule
\end{tabular}
\label{right_turn_steps}
\caption{right turn, steps}
\end{table}


